Wie man zitiert. Für indirekte Zitate setzt man am Ende ganz einfach den Snippet: \begin{verbatim}
\cite[]{anderie2018gamification}
\end{verbatim}
Selbst wenn keine Seitenanzahlen angegeben werden, müssen die einfachen Klammern nach dem cite-Befehl gesetzt werden, weil ein zusätzliches Package (natbib) eingebunden ist, welches passive Autorenzitierung erlaubt \cite[]{anderie2018gamification}. Für das Einbinden der Seitenanzahl, kann einfach in die eckigen Klammern geschrieben werden, z.B. \anf{S.\textasciitilde10f.}, dabei sorgt die Tilde für einen gleichbleibenden Abstand des Leerzeichens \cite[S.~10f.]{anderie2018gamification}. 
\par 
Wird über mehrere Sätze hinweg zitiert kann ein Trick angewendet werden. Dabei werden die Autoren direkt in die Sätze mit eingebunden und deren Aussagen über mehrere Sätze hin verknüpft. Beispiel:\par 
\cite{Stef12} beschreiben in ihrer Arbeit drei grundsätzliche Konzepte, die allesamt eine Rolle in der Gamifizierung spielen. Die Autoren weisen darauf hin, dass vor allem das Konzept A die größte Bedeutung hat. Dennoch sind sie der Meinung, dass es in der nahen Zukunft zu einer großen Veränderung kommen wird, die sich hauptsächlich durch Grund B begründet lässt (S.~25ff.).
\par 
Diese Art der Zitierung kann erreicht werden, indem die eckigen Klammern entfernt werden:
\begin{verbatim}
\cite{Stef12}
\end{verbatim}
Die referenzierten Seiten werden einfach per Hand am Ende des Absatzes hinzugefügt.
\par
Bilder und Kapitel referenziert man ganz einfach mit:
\begin{verbatim}
\ref{fig:hslogoRight} 
\ref{sec:mathe}

Sind die Bezeichnungen der Label.
\label{fig:hslogoRight}
\end{verbatim}
In Abbildung \ref{fig:hslogoRight} erkennt man...
\par 
Im Kapitel \ref{sec:sonstiges} wird beschrieben, dass